%% ----------------------------------------------------------------
%% Thesis.tex
%% ---------------------------------------------------------------- 
%% Final copy must be double sided printing.
\documentclass[twoside]{ecsthesis}      % Use the Thesis Style
\graphicspath{{figures/}}   % Location of your graphics files
\usepackage{natbib}            % Use Natbib style for the refs.

%% \removecolourlinks    % Uncomment this command to remove colour from any links


\input{Definitions}            % Include your abbreviations
%% ----------------------------------------------------------------
\begin{document}
\frontmatter
\title      {Spatial and scale dependence of a taxonomic-structured biodiversity index and its relation to climatic variables }
\authors    {\texorpdfstring
             {\href{mailto:jmem1a13@soton.ac.uk}{Juan Escamilla}}
             {Juan M. Escamilla M\'olgora}
            }
\department  {Geography and the Environment}
\group       {GEM Masters Course}
\addresses  {\groupname\\\deptname\\\univname}
\date       {\today}
\subject    {}
\keywords   {biodiversity, phylogeny, scale, correlation, climate}
\maketitle
\begin{abstract}
Biodiversity indices have been widely used as indicators of ecosystem stability and composition. It has been recently demonstrated that species and phylogenetic diversity could be combined using different spatial scales to show the historical, deterministic and stochastic nature of processes that affect the compositions of ecosystems’ assemblage of species. In this work I propose the use of the taxonomic classification to define a phylogenetic-based biodiversity indexed tree structure.  This structure integrates information at the levels of: kingdom, phylum, order, class, family, genus and species as well as the mathematical properties inherited by network theory.  The data used for this work are the occurrences of species in the Global Biodiversity Information Facility (GBIF) and the Global Climate Data (worldclim.org).
The main objective is to determine at what extent the index varies across scales within a biome and how is the variation with respect to climatic variables. A simple abundance-based diversity index will be used in each taxonomic level. After this, a continuous point-referenced abundance data raster model (CPRARM) will be generated for each taxonomic level disregarding, for now, the multiple sampling frames and designs in the dataset.  The variations in space as well as the fractal dimension will be used in each taxonomic-level CPRARM to assess the space and scale dependencies. The proposed methodology will be used to characterize taxonomic-structured index families (trees) that remain invariant within a biome (i.e. the ranges and shapes of trees that occur in a given biome.). These results will give details that can answer questions about the distribution of the phylogenetic structures through different biomes, the fragmentation and symmetry of these structures through scale and space, and the loss of species and phylogenetic diversity driven by human impacts.

\end{abstract}
\tableofcontents
\listoffigures
\listoftables


%% -----------------------
%% Authorship declaration
%% -----------------------
%% Either include citations like below (as many as required spaced with commas or 'and').
\authorshipdeclaration{\citep{Gunn:2001:pdflatex}, \citep{Lovell:2011:updated} and \citep{Gunn:2011:updated2}}
%% Or state no citations like below
%% \authorshipdeclaration{}
%% -----------------------

\acknowledgements{Thanks to no one.}
%\listofsymbols{ll}{$w$ & The weight vector}
\mainmatter
%% ----------------------------------------------------------------
\include{Introduction}
\include{Chap1}
\include{Conclusions}
\appendix
\include{AppendixA}
\backmatter
\bibliographystyle{apalike}
\bibliography{ECS}
\end{document}
%% ----------------------------------------------------------------
